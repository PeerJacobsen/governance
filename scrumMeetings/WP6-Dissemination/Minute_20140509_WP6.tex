\documentclass[a4paper, 11pt]{article}
\usepackage[ascii]{inputenc}
\usepackage{supertabular}
\usepackage[ngerman]{babel}
\usepackage{amsmath}
\usepackage{amssymb,amsfonts,textcomp}
\usepackage {geometry}
\geometry{a4paper,top=25mm,left=30mm,right=25mm,bottom=30mm}
\usepackage{color}
\usepackage{array}
\usepackage{hhline}
\usepackage{hyperref}
\hypersetup{colorlinks=true, linkcolor=blue, citecolor=blue, filecolor=blue, urlcolor=blue}


\begin{document}
{\begin{center}\huge\bf openETCS WP6 Meeting Minutes\end{center}}
\section{Meeting Information}

\renewcommand{\arraystretch}{1.5}
\begin{supertabular}{m{.2\textwidth}m{.8\textwidth}}
%\hline
Subject & WP6 Weekly Scrum: Dissemination\\
Date \& time & 2014-05-09, 10:15h--10:30h\\
Location & Telco and Goto-Meeting\\
Called up by & Stefan Rieger\\
Participants &
%Arnaud,
%Sylvain Baro,
%Benjamin Beichler,
%Marc Behrens,
%Cecile Braunstein,
Fausto Cochetti,
Lukasz Fronc,
%Klaus R\"udiger Hase,
%Frank Golatowski,
%Hardi Hungar,
%Baseliyos Jacob
%Michael Jastram,
Peter Mahlmann,
Alexander Nitsch,
%Matthieu Perin,
%Marielle Petit-Doche,
Stefan Rieger,
Uwe Steinke,
%Izaskun de la Torre,
%Silvano dal Zilio,
Jan Welte
%Giovanni Zanelli
\\

Minutes by & Bernd Hekele\\

%\hline
\end{supertabular}
\renewcommand{\arraystretch}{1.0}

%\line(1,0){440}

\section{Agenda}
\begin{itemize}

\item Innotrans

\end{itemize}

\section{Discussion}

\begin{itemize}
\item Dissemination Activity Report\\
The update of the dissemnation repost is due. Indication your publications and other activities.
 
\item openETCS @ InnoTrans
openETCS will be present at the DB stand at the Innotrans. openETCS has access to the speakers' corner.

The follow-up of the openETCS @ Innotrans activities is visible on Github:
\url{https://github.com/openETCS/Dissemination/wiki/INNOTRANS-2014-Planning}. A doodle is availble for you to indicate you participation. If you plan to hold a presentation your are supposed to make a proposal on the wiki or send a mail to Stefan.

\end{itemize}

%\line(1,0){440}
\section{Notes}

\end{document}
